% easychair.tex,v 3.2 2012/05/15
%
% Select appropriate paper format in your document class as
% instructed by your conference organizers. Only withtimes
% and notimes can be used in proceedings created by EasyChair
%
% The available formats are 'letterpaper' and 'a4paper' with
% the former being the default if omitted as in the example
% below.
%
\documentclass{easychair}
%\documentclass[debug]{easychair}
%\documentclass[verbose]{easychair}
%\documentclass[notimes]{easychair}
%\documentclass[withtimes]{easychair}
%\documentclass[a4paper]{easychair}
%\documentclass[letterpaper]{easychair}

% This provides the \BibTeX macro
\usepackage{doc}
\usepackage{makeidx}

% In order to save space or manage large tables or figures in a
% landcape-like text, you can use the rotating and pdflscape
% packages. Uncomment the desired from the below.
%
% \usepackage{rotating}
% \usepackage{pdflscape}

% If you plan on including some algorithm specification, we recommend
% the below package. Read more details on the custom options of the
% package documentation.
%
% \usepackage{algorithm2e}

% Some of our commands for this guide.
%
\newcommand{\easychair}{\textsf{easychair}}
\newcommand{\miktex}{MiK{\TeX}}
\newcommand{\texniccenter}{{\TeX}nicCenter}
\newcommand{\makefile}{\texttt{Makefile}}
\newcommand{\latexeditor}{LEd}

%\makeindex

%% Document
%%
\begin{document}

%% Front Matter
%%
% Regular title as in the article class.
%
\title{Digital Repository of Mathematical Formulae}

% \titlerunning{} has to be set to either the main title or its shorter
% version for the running heads. When processed by
% EasyChair, this command is mandatory: a document without \titlerunning
% will be rejected by EasyChair

\titlerunning{The {\easychair} Class File}

% Authors are joined by \and. Their affiliations are given by \inst, which indexes
% into the list defined using \institute
%
\author{
Howard S. Cohl\inst{1}
\and
    Deyan Ginev\inst{2}
\and
   Marje A.~McClain\inst{1}
\and
   Bonita V.~Saunders\inst{1}
\and 
   Moritz Schubotz\inst{3}
\and 
   Janelle C.~Williams \inst{4}
}

% Institutes for affiliations are also joined by \and,
\institute{Applied and Computational Mathematics Division,\\
National Institute of Standards and Technology,
Gaithersburg, Maryland, U.S.A.\\
\email{howard.cohl@nist.gov, marjorie.mcclain@nist.gov, bonita.saunders@nist.gov}
\and
School of Engineering and Science,
Jacobs University, Bremen, Germany\\
\email{d.ginev@jacobs-university.de}\\
\and
Database Systems and Information Management Group,\\
Department of Software Engineering and Theoretical Computer Science,\\
Technische Universit\"{a}t, Berlin, Germany\\
\email{schbotz@tu-berlin.de}\\
\and
Department of Mathematics and Computer Science,\\
Virginia State University, Petersburg, VA, U.S.A.\\
\email{janelle.williams35@gmail.com}\\
}
\authorrunning{Cohl, Ginev, McClain, Saunders, Schubotz and Williams} 
% has to be set for the shorter version of the authors' names;
% otherwise a warning will be rendered in the running heads. When processed by
% EasyChair, this command is mandatory: a document without \authorrunning
% will be rejected by EasyChair

%\authorrunning{Mokhov, Sutcliffe, Voronkov and Gough}


\clearpage

%%%%%%%%%%%%%%%%%%%%%%%%%%%%%%%%%%%%%%%%%%%%%%%%%%%
\maketitle
%%%%%%%%%%%%%%%%%%%%%%%%%%%%%%%%%%%%%%%%%%%%%%%%%%%

\begin{abstract}
The purpose of the Digital Repository of Mathematical Formulae (DRMF) is to 
create a digital knowledge base of mathematical formulae for orthogonal polynomials 
and special functions (OPSF) and of associated mathematical data.  The DRMF 
addresses two separate needs of working mathematicians, physicists and engineers: 
providing the technical infrastructure to publish and interact with OPSF formulae
on the web, as well as the editorial insight to successfully curate such a resource.  
Using MediaWiki, the DRMF builds on top of previous efforts, adapting their technology 
to support OPSF as a scientific web domain.  Whereas PlanetMath and Wikipedia expose 
concepts or terms as first-class objects, both from a system and authoring perspective, 
the DRMF does that with mathematical formulae.
\end{abstract}

%\setcounter{tocdepth}{2} {\small \tableofcontents}

%\section{To mention}
%
%Processing in EasyChair - number of pages.
%
%Examples of how EasyChair processes papers. Caveats (replacement of EC
%class, errors).

%\pagestyle{empty}


%------------------------------------------------------------------------------
\section{Introduction}
\label{sect:introduction}

Compendia of mathematical formulae have a long and rich history. Many
scientists have developed such repositories as books and these have been
extremely useful to scientists, mathematicians and engineers over the last 
several centuries (see 
\cite{
Brych,
ErdelyiHTF,
ErdelyiTIT,
Grad,
MOS,
PrudBrychMar}
for instance).  
There has been much overlap in formulae for the different 
compendia, but 
useful specific information has often been captured in 
single references and one often would need to be familiar with many different
compendia to find to a desired formula.  Online compendia of mathematical 
formulae exist such as voluminous Wolfram Functions Site 
(\url{http://functions.wolfram.com/}), subsets of Wikipedia 
(\url{http://en.wikipedia.org}), and the NIST Digital Library of Mathematical
Functions (\url{http://dlmf.nist.gov/}).  We hope to take advantage of the
best aspects of these online efforts and to take advantage of powerful new
features which a community arm of scientists should find beneficial.
Ingredients which we would like the online repository to incorporate include: 
\begin{itemize}
\item the ability to interact with a community of mathematicians and scientists 
who may enter formulae data relevant to their own research related to OPSF 
(as Wikis easily allow),
\item should not be limited in its description 
to an ``encyclopedic'' viewpoint, i.e.~understandable only to those who are 
members of the general public, 
\item should be internally understandable in a standalone fashion
(for instance by consistently using extended \LaTeX ML DLMF (eDLMF) macros to define 
special functions to allow for easy access to definitions and to facilitate 
cross-repository search),
\item should not be limited in size as in a printed book,
\item should have a user friendly and consistent viewpoint and authoring perspective,
\item should take advantage of powerful modern MathML tools for easy to read rendered 
mathematics at different font sizes, 
\item should have the ability to link to currently existing online resources.
\end{itemize}
Note that we have summarized our mission requirements in an online document
(\url{http://drmf.instance-proxy.wmflabs.org/wiki/DRMF_Requirements}).
It is the desire of our group to build a tool which 
attempts to bring the above features together to be a tool for mathematicians 
and scientists to bridge together currently existing compendia and to publish new orthogonal 
polynomial and special function (OPSF) formulae, and their corresponding 
mathematical data, online and on the web.  We will refer to this web tool as a
Digital Repository of Mathematical Formulae (DRMF).

An origination of this concept can be ascertained by viewing a Society for
Industrial and Applied Mathematics activity group OPSF-Net post by Dmitry Karp 
(2011) \cite[Topic \#5]{OPSFNET18_4}.  
In that OPSF-Net edition, there were two related 
posts \cite[Topic \#6, Topic\#7]{OPSFNET18_4} with a follow-up post in
\cite[Topic \#3]{OPSFNET18_6}.

\section{Implementation}

On the technical front, we have built on the experience of the 
DLMF project \cite[see also \url{http://dlmf.nist.gov}]{NIST}, as well as the 
Planetary (\url{http://planetmath.org}) and 
MediaWiki (\url{http://www.mediawiki.org}) publishing platforms. 
Each feature of the publishing process is modeled 
as a separate general-purpose MediaWiki extension, and published on GitHub 
(\url{https://github.com}) for 
redistribution.  The first demonstration provides experimental extensions 
for (1) formula interactivity, (2) formula home pages, (3) centralized bibliography 
and (4) mathematical search. At its foundation, DRMF shares the core technologies 
of the DLMF, based on community-recognized open standards (\TeX, 
HTML+MathML), as 
stringed together by the \LaTeX ML (\url{http://dlmf.nist.gov/LaTeXML/}) converter. We incorporate some of the high-quality 
components used in Planetary: the JOBAD interactivity framework
(\url{https://github.com/KWARC/jobad}), MathWebSearch search engine
(\url{http://search.mathweb.org/}), as well as those designed for 
MediaWiki: the MathSearch 
(\url{http://www.mediawiki.org/wiki/Extension:MathSearch})
and Math extension
(\url{http://www.mediawiki.org/wiki/Extension:Math}).  
We will investigate the use of MathJax and JOBAD for menues and formula interactivity 
mechanisms that we hope to incorporate. 

Examples of formula 
interactivity include:
\begin{itemize}
\item a clipboard for mathematics, allowing easy copy/paste of a formula's source, presentation or content representations;
\item on-demand summary of constraints and substitutions applicable to a formula;
\item on-demand consulting with external web services 
(e.g., WolframAlpha (\url{https://www.wolframalpha.com/})) 
and local computer algebra systems (e.g., Mathematica \cite{Mathematica}, 
Maple (\url{http://www.maplesoft.com}), Sage
(\url{http://www.sagemath.org/})).
\end{itemize}

The DRMF's first vision of a repository for mathematical formulae is to treat any 
notable expression as a primary object of interest, describing it in its 
own formula home page. Currently, formula home pages contain: (1) a description of 
the formula itself, (2) open section for proofs, (3) bibliographic citations, 
and (4) a glossary of special extended DLMF symbols and \LaTeX ML extended DLMF 
(eDLMF) macros used in the formula with links to 
definitions, and (5) a link to the formula in the DLMF, whenever applicable.  
Optionally, one may also enter relevant constraints, 
substitutions, notes, the formula name, as well as links to formula generalizations 
and specializations.  For each formula home page there is a corresponding talk page 
where discussions about the formula and its page may take place. We are 
incorporating a strategy for handling the insertion of formula errata.

We are also exploring a variety of search strategies within our MediaWiki
deployment. A key asset in our development of search capabilities is the use of 
eDLMF macros in writing the \TeX\,\,markup of formulae. \LaTeX ML uniquely translates the 
macros for specific special functions, orthogonal polynomials and general 
mathematical objects into Content MathML symbols, which in turn fuel structural 
search engines, such as MathWebSearch, to return results with very high accuracy.

Next, we present an early overview of the shortlisted seed resources which we plan 
to incorporate. We have been given permission to seed the DRMF with data from the 
NIST DLMF \cite{NIST}.  We would also like to extend the DLMF list of formulae by
including all relevant formulae which are cited within the DLMF.  We have also been
given permission to include formulae data from Koekoek, Lesky and Swarttouw's (KLS) 
book \cite{Koekoeketal}.  We also plan to incorporate Tom Koornwinder's 
additions to KLS which are given in \cite{KoornwinderKLSadd}.
We have also been given permission to incorporate seed formula data from the
Bateman Manuscript Project (BMP) Higher Transcendental Functions and Tables 
of Integral Transforms \cite{ErdelyiHTF,ErdelyiTIT}.  Efforts to 
upload BMP data are known to be extremely difficult since this effort will rely 
on the use of mathematical optical character recognition software to produce 
\LaTeX\,\,source for these formulae. This effort is currently under consideration
for feasibility of use.
We are in active communication with other publishers to get permission for other 
proven sources for mathematical formulae.  

Our current focus is on seeding the 
DRMF with DLMF data, and we have completed this for Chapter 25 in the DLMF entitled
Zeta and Related Functions.  Future near-term efforts will focus on seeding the rest of 
the DLMF data as well as the KLS data with eDLMF macros incorporated.  


\section*{Acknowledgements}

We would like to express our deep gratitude to the KWARC group at Jacobs University,
Bremen, Germany, and especially to its group leader, Michael Kohlhase, for allowing 
us to use their server in the initial stages of our DRMF development.  We would
also like to thank Dan Lozier, Tom Koornwinder, Dmitry Karp, Dan Zwillinger, 
Victor Moll, Bruce Miller, and Hans Volkmer for offering their advice and for
valuable discussions.

%------------------------------------------------------------------------------
% Refs:
%
\label{sect:bib}
\bibliographystyle{plain}
%\bibliographystyle{alpha}
%\bibliographystyle{unsrt}
%\bibliographystyle{abbrv}
\bibliography{/home/hcohl/tex/refbib}

%------------------------------------------------------------------------------
%\appendix
%\section{{\easychair} Requirements Specification}
%\label{sect:easychair-requirements}

%\begin{figure}
%  \begin{centering}
%    \includegraphics[width=0.89\textwidth]{debug.pdf}
%    \caption{A page of a document created using the \texttt{debug}
%      option} 
%    \label{fig:redframe}
%  \end{centering}
%\end{figure}


%------------------------------------------------------------------------------
% Index
%\printindex

%------------------------------------------------------------------------------
\end{document}

% EOF
